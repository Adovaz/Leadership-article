%---------------------------------
%Basic Geometry, Encoding and Font
%---------------------------------
\documentclass[british,12pt,a4paper]{article}
\usepackage{helvet}
\renewcommand\familydefault{\sfdefault}
\usepackage[T1]{fontenc}
\usepackage[left=2.54cm, right=2.54cm, top=2.54cm, bottom=2.54cm]{geometry}
\usepackage{csquotes}
%------------
%Bibliography
%------------
\usepackage[
backend=biber,
style=apa,
natbib=true,
]{biblatex}
\addbibresource{Leadership Article.bib}
%--------------
%Setting Title
%--------------
\title{{\fontsize{14}{15} \textbf{Discharge Medication Education}}\vspace{-1em}}
\author{{\fontsize{14}{15} \textbf{Student ID's: \textit{326609248, 915365341 and 997283760 }}}}
\date{}
%----------------------
%Section heading style
%----------------------
\usepackage{titlesec}
\titleformat{\section}
{\centering\normalfont\fontsize{14}{15}\bfseries}{\thesection.}{1em}{}
\setcounter{secnumdepth}{0}
\titlespacing\section{0pt}{12pt plus 4pt minus 2pt}{2pt plus 2pt minus 2pt}
\titleformat{\subsection}
{\normalfont\fontsize{14}{15}\bfseries}{\thesection.}{1em}{}
\setcounter{secnumdepth}{0}
\titlespacing\subsection{0pt}{12pt plus 4pt minus 2pt}{2pt plus 2pt minus 2pt}
\titleformat{\subsubsection}
{\normalfont\fontsize{14}{15}\bfseries\itshape}{\thesection.}{1em}{}
\setcounter{secnumdepth}{0}
\titlespacing\subsubsection{0pt}{12pt plus 4pt minus 2pt}{2pt plus 2pt minus 2pt}
%------------
%Header Style
%------------
\usepackage{fancyhdr}
\pagestyle{fancy}
\fancyhead{}
\fancyhead[L]{DISCHARGE MEDICATION EDUCATION}
\fancyhead[R]{}
\fancyfoot{}
\fancyfoot[R]{\thepage}
\setlength{\headheight}{15pt}
\renewcommand{\headrulewidth}{0pt}
%--------------------------------------
%Indenting and Paragrapth justification
%--------------------------------------
\usepackage{setspace}
\usepackage{parskip}
\newlength{\normalparindent}
\setlength{\normalparindent}{\parindent}
\raggedright
\setlength\parskip{5pt}
\setlength\parindent{0.5 in}
\setlength\bibhang{0.5 in}
\usepackage{indentfirst}
\doublespacing
%----------------------------
%Coloring for links and table
%----------------------------
\usepackage{xcolor}
\usepackage{colortbl}
\usepackage{multirow}
%-------------
%Table spacing
%-------------
\setlength{\tabcolsep}{10pt}
\renewcommand{\arraystretch}{1.5}
%-------------------------
%Hyperlinks and PDF set up
%-------------------------
\usepackage{hyperref}
\hypersetup{
	colorlinks=true,      
	urlcolor=blue,
	citecolor=black,
	linkcolor=black,
	pdfauthor={Avery Zavoda},
	pdftitle=Nursing,
	pdfsubject=Nursing,
	pdfproducer=Latex with hyperref,
	pdfpagemode=FullScreen,
	pdfcreator=pdfLatex,
}
%--------------
%Begin Document
%--------------
\begin{document}
	%--------------------------------
	%Print Title and Initiate Heading
	%--------------------------------
	\maketitle
	\thispagestyle{fancy}
	%---------
	%Main Body
	%---------
	<<Content>>
	\section{Background}
	The transition from hospital care to discharge poses a significant risk factor for medication-related harm to occur \parencite{Flatman2021}. \textcite{Alqenae2020} highlighted that post discharge, over 50\% of adult patients encounter medication errors and nearly 20\% experience an adverse drug event. Medication education prior to discharge is pivotal in preventing errors, and ensuring the patient feels competent in self-administering the prescribed medication regime to avoid exacerbation of the patient's medical conditions \parencite{Hajialibeigloo2021}. Upon admission, patients' medication regimes have potential to change substantially by discharge, requiring readjustment to this new regime \parencite{Weir2020}. 

	During hospitalisation, healthcare professionals and most typically the nurse are responsible for medication management which upon discharge, responsibility is transferred to the patient with limited external guidance from healthcare professionals \parencite{Mortelmans2021}. \citeauthor{Mortelmans2021} also depicts the knowledge required by the patient to safely administer and follow the medication regime post-discharge including knowing and recognising their medications, the medication schedule, and the ability to administer the correct dose at the correct time. 

	As experienced by students on placement, the nurse taking the time to explain medication administration and regimes to the patient prior to discharge is extremely beneficial to patient confidence and capabilities of self-managing, and adherence to the prescribed medication regimes. For example, a student and RN spoke to a patient prior to discharge feeling anxious and stressed by the medications prescribed for them to follow. The student and the RN took the time to explain to the patient and whanau identifying characteristics and the purposes of each medication, when it was required to be taken and how each medication was prescribed in the regime. The patient expressed genuine gratitude to the student and RN for educating and communicating the requirements of their medication regime, stating as they can comprehend what is required, they feel more competent to identify and adhere to what is required of them. 

	\section{Discussion}
	<<Content>>
	\section{Conclusion}
	<<Content>>
	\section{Recommendations}
	<<Content>>
	%---------
	%Word count
	%---------
	\vfill
	\begin{center}
		\large Word Count: 
	\end{center}
	%---------
	%Print  Bibliography
	%---------
	\pagebreak
	\doublespacing
	\printbibliography
	\end{document}
