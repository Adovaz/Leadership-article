%---------------------------------
%Basic Geometry, Encoding and Font
%---------------------------------
\documentclass[british,12pt,a4paper]{article}
\usepackage{helvet}
\renewcommand\familydefault{\sfdefault}
\usepackage[T1]{fontenc}
\usepackage[left=2.54cm, right=2.54cm, top=2.54cm, bottom=2.54cm]{geometry}
\usepackage{csquotes}
%------------
%Bibliography
%------------
\usepackage[
backend=biber,
style=apa,
natbib=true,
]{biblatex}
\addbibresource{Leadership Article.bib}
%--------------
%Setting Title
%--------------
\title{{\fontsize{14}{15} \textbf{Discharge Medication Education}}}
\author{{\fontsize{14}{15} Student ID's: 289955511, 326609248, 876923194,} \\ {\fontsize{14}{15}902467787, 915365341 and 997283760}}
\date{}
%----------------------
%Section heading style
%----------------------
\usepackage{titlesec}
\titleformat{\section}
{\centering\normalfont\fontsize{14}{15}\bfseries}{\thesection.}{1em}{}
\setcounter{secnumdepth}{0}
\titlespacing\section{0pt}{12pt plus 4pt minus 2pt}{2pt plus 2pt minus 2pt}
\titleformat{\subsection}
{\normalfont\fontsize{14}{15}\bfseries}{\thesection.}{1em}{}
\setcounter{secnumdepth}{0}
\titlespacing\subsection{0pt}{12pt plus 4pt minus 2pt}{2pt plus 2pt minus 2pt}
\titleformat{\subsubsection}
{\normalfont\fontsize{14}{15}\bfseries\itshape}{\thesection.}{1em}{}
\setcounter{secnumdepth}{0}
\titlespacing\subsubsection{0pt}{12pt plus 4pt minus 2pt}{2pt plus 2pt minus 2pt}
%------------
%Header Style
%------------
\usepackage{fancyhdr}
\pagestyle{fancy}
\fancyhead{}
\fancyhead[L]{DISCHARGE MEDICATION EDUCATION}
\fancyhead[R]{}
\fancyfoot{}
\fancyfoot[R]{\thepage}
\setlength{\headheight}{15pt}
\renewcommand{\headrulewidth}{0pt}
%--------------------------------------
%Indenting and Paragrapth justification
%--------------------------------------
\usepackage{setspace}
\usepackage{parskip}
\newlength{\normalparindent}
\setlength{\normalparindent}{\parindent}
\raggedright
\setlength\parskip{5pt}
\setlength\parindent{0.5 in}
\setlength\bibhang{0.5 in}
\usepackage{indentfirst}
\doublespacing
%----------------------------
%Coloring for links and table
%----------------------------
\usepackage{xcolor}
\usepackage{colortbl}
\usepackage{multirow}
%-------------
%Table spacing
%-------------
\setlength{\tabcolsep}{10pt}
\renewcommand{\arraystretch}{1.5}
%-------------------------
%Hyperlinks and PDF set up
%-------------------------
\usepackage{hyperref}
\hypersetup{
	colorlinks=true,      
	urlcolor=blue,
	citecolor=black,
	linkcolor=black,
	pdfauthor={Avery Zavoda},
	pdftitle=Nursing,
	pdfsubject=Nursing,
	pdfproducer=Latex with hyperref,
	pdfpagemode=FullScreen,
	pdfcreator=pdfLatex,
}
%--------------
%Begin Document
%--------------
\begin{document}
%--------------------------------
%Print Title and Initiate Heading
%--------------------------------
\maketitle
\thispagestyle{fancy}
%---------
%Main Body
%---------
The fundamentals of care framework encompass a holistic approach to patient care, aiming to address a patient's physical, psychosocial and relational well-being \parencite{Dempsey2013}.  The framework's components include relationship building, integration of care, and context of care. Integration of care is a significant component of the fundamentals of the care framework.

The specific example this article will discuss involves the importance of medication education for patients upon discharge. The chosen example is relevant to this framework as the integration of care includes factors such as providing patient education and information. Implementing this into nursing care helps address specific patient needs, such as psychosocial needs.

Medication education upon discharge facilitates informed decision-making and empowers individuals to participate actively in their health management. By providing patients with knowledge about their conditions, treatment plans, and self-care strategies, nurses enable them to take ownership of their health and make health-promoting choices. Patient education also fosters patient confidence. When patients know what to expect and feel confident in their understanding of their medications, they are less likely to experience anxiety or stress related to their treatment. Furthermore, educating patients with comprehensive information, including dosage, administration techniques, and potential side effects, reduces the likelihood of medication-related errors post-discharge.

This article will explore the relationships between the fundamentals of care framework, patient education, and the crucial role of medication education in ensuring patient safety and optimising health outcomes.


\section{Background}
The transition from hospital care to discharge poses a significant risk factor for medication-related \parencite{Flatman2021}. \textcite{Alqenae2020} highlighted that post-discharge, over 50\% of adult patients encounter medication errors, and nearly 20\% experience an adverse drug event. Medication education before discharge is pivotal in preventing errors and ensuring the patient feels competent in self-administering the prescribed medication regime to avoid exacerbating the patient's medical conditions \parencite{Hajialibeigloo2021}. After admission, patients' medication regimes have the potential to change substantially by discharge, requiring readjustment to this new regime \parencite{Weir2020}.

Healthcare professionals, typically nurses, are responsible for medication management during hospitalisation. Upon discharge, responsibility is transferred to the patient with limited external guidance from healthcare professionals \parencite{Mortelmans2021}. \textcite{Mortelmans2021} also depicts the knowledge required by the patient to safely administer and follow the medication regime post-discharge, including knowing and recognising their medications, their medication schedule, and the ability to administer the correct dose at the proper time.

As experienced by students on placement, the nurse taking the time to explain medication administration and regimes to the patient before discharge benefits patient confidence, self-managing abilities, and adherence to the prescribed medication regimes. For example, a student and RN spoke to a patient before discharge, feeling anxious and stressed by the medications prescribed for them to follow. The student and the RN took the time to explain to the patient and whanau identifying characteristics and the purposes of each medication, when it was required to be taken and how each medication was prescribed in the regime. The patient expressed genuine gratitude to the student and RN for educating and communicating the requirements of their medication regime, stating that as they can comprehend what is required, they feel more competent to identify and adhere to what is needed.

\section{Discussion}
\textcite{Phatak2015} found that pharmacist involvement and medication education when discharging a patient decreased the incidence of patients being readmitted to the ward due to relapse or their condition worsening. It was also found that adequately explaining the medication to the patient decreased ED visits. \citeauthor{Phatak2015} states that adverse drug events paired with patients being uneducated about the medications prescribed post-discharge are leading factors in readmission. Similarly, \textcite{Alper2023} found in a meta-analysis that medication review by the nurse to the patient is likely to reduce hospital readmission by reducing the likelihood of an adverse drug event occurring. This shows the importance of education before discharge, as hospitals are busy environments; the readmission of patients by preventable circumstances adds to the stress of the healthcare professionals and negatively impacts the patient's physical and psychosocial well-being. \citeauthor{Phatak2015} found "that less than 60\% of patients knew the indication for a new medication prescribed at discharge". This shocking statistic shows the worth of taking time to educate the patient. Understanding the role and purpose of the medication gives patients a sense of agency and prevents unnecessary adverse drug reactions.

\citeauthor{Alper2023} states that medication education "provides an opportunity for clinicians to ensure that patients understand what medications they are taking, how to take them, and why they are taking them". The systematic review evaluates that medication education is associated with a decrease in actual and potential adverse drug events. When a patient leaves the hospital, although it may be the end of their journey with the nurses and healthcare professionals, their medical journey continues for the individual and their whanau. Whether the discharged patient is required to take medications for the rest of their lives, for only a short period, or as needed, they must get the education to develop a proper understanding of the medications they are taking so that they can take them safely \parencite{Yap2016}.

Aiding in the discharge of patients and explaining their discharge summary and medication regime are all part of a nurse's role. \textcite{Sulosaari2014} depicts medication management as one of the fundamental roles of the nurse and the importance of ensuring sufficient medication competence to the patient prior to discharge. Although other healthcare professionals, such as pharmacists and doctors, are responsible for ensuring that the person getting discharged is educated appropriately, the relationship between the nurse and patient allows for information to be conveyed more comprehensively. The established trusting and interpersonal relationship between the patient and nurse throughout their stay allows the nurse to be the person the patient will go to with questions that the nurse can answer competently. Often, on clinical placements, students have found that the patient may understand what is being explained to them but have further questions later. This is perfectly normal, and often, the patient will ask the nurse when this situation arises. It is then essential that the RN takes time to fully explain the medication and all of the relevant information regarding that specific medication \parencite{Flatman2021}.


\section{Conclusion}
Medication education is a critical component within the fundamentals of care framework as it significantly impacts the outcomes and safety of a patient. This interconnects with the framework's holistic approach to patient care, which integrates physical, psychosocial and relational well-being. The transition from hospital to home care poses a high risk of medication-related harm. However, studies indicate that comprehensive patient education during this discharge transition can effectively reduce medication errors and adverse drug events. The role of nurses in medication education is crucial. Nurses educate the patient during discharge by explaining medication regimens and discharge summaries, further developing a trusting relationship with the patient and their caregivers during and after the hospital stay. This relationship with the patient allows the nurse to deliver information comprehensibly and answer any possible questions the patients may have, ensuring they fully understand their medication instructions.

Studies by \citeauthor{Alper2023} and \citeauthor{Phatak2015} focus on the subject of thorough medication education lowering the risk of hospital and emergency department readmissions, which emphasises the importance of medication education in patient care. Including family members and caregivers during education can enhance knowledge, adherence, and confidence in medication administration. As \citeauthor{Alper2023} stated, medication education ensures that patients understand the medication they are administering, how to take it, and why they are taking it. This knowledge is critical for the patient's health journey, as well as their caregivers and whanau, especially after discharge.

Medication education is an essential part of the fundamentals of care framework, especially for optimising health outcomes, ensuring patient safety and supporting a comfortable transition from hospital to community care. This approach helps the patient's well-being and reduces the burden on healthcare systems by preventing unnecessary readmissions and promoting effective long-term health management.

\section{Recommendations}
When providing recommendations, we want change to align positively with 'The Fundamentals of Care Framework' \parencite{Dempsey2013} because, as nurses, the trusting relationship we establish with patients is what will aid in us educating a patient well and altering the information given to address their needs. Patients often feel comfortable directing their questions to the nurses involved, and usually, the nurse spends more time with the patient than other health professionals and, therefore, can accommodate and tailor answers to help the patient understand their medications. Issues around education upon discharge on medication discussed fall towards the medical competency nurses have regarding the medications, how they allow time to discuss information with patients and their families and the systematic way nurses deliver discharge education.

Concerning medication competency amongst nurses, programmes such as graduate support and education workshops could provide a safety net to those nurses feeling they need more preparation for the new environment they are now working in. \citeauthor{Sulosaari2014} states that there was room for improvement through policy and educational initiatives based on their study of the education provided in nursing schools for medication. Whilst this is for a nursing school level, we recommend that similar initiatives, policies and guidelines be implemented in the workplace and at a postgraduate level, as learning should continue after graduation. Initiatives could be mini workshops nurses can attend each week, which speak to a different medication or aspect of clinical knowledge. Healthcare strives to find new ways to provide care for patients and search for and achieve better health outcomes for patients. Proper medication education on discharge is a small part that could change the success of patients and their conditions long-term within the community. This change would also empower the nurse to feel confident in advising on discharge medication and allow the patient consistency in care.

Time is another constraint for nurses, as hospitals can be busy environments to work in. However, the time allowed for discussions with patients pre-discharge is crucial to their success in the community \parencite{Sanjai2019}. With time, nurses cannot only discuss but also help teach patients how to self-medicate their medications whilst still in the hospital. \citeauthor{Mortelmans2021} state that during hospitalisation, most of the medication administration is done by health care professionals, and this leaves patients feeling unprepared for self-medication post-discharge. If patients can practise self-administering medication with a nurse whom they trust and who is able to support them psychosocially through encouragement, this builds confidence for the patient that they can overcome further medical challenges they may face, which can be a big fear \parencite{Sanjai2019}. On placement, students saw that patients they observed self-medicating with their preceptor were able to ask questions that might arise at home and have them answered in the safety of the hospital environment by a health professional who was able to guide the patient and provide education in a way that tailored to their understanding which would overall reduce readmission \parencite{Sanjai2019}. We recommend that more time be allowed for nurses to utilise teaching patients how to draw up and administer their medications. This change has the potential to significantly reduce the number of readmissions, adverse drug events and non-compliance with medication.

Students on placement found that patients expressed that it was helpful and easier to understand and comply with medication when it was explained to them and provided with a physical reminder and explanation. Everyone is different, and we need to remember that to achieve patient-centred care, it not only starts with us building and forming a trusting relationship but also relies on how well we know our patient and whether we can individualise the care we give to their specific needs as they have a right to quality care that considers their individuality \parencite{NZNC2012}. \textcite{Sanjai2019} conducted a study on whether an educational video on medication upon discharge would benefit patients' and nurses' experiences at discharge and understanding and confidence post-discharge. The study concluded that the educational video significantly improved patients' feelings towards preparation for discharge. While this method of education may work for some, we recommend a mixed approach. We would also recommend that flyers or handouts that have all the information written down be given to patients, as due to fear and anxiety around hospitalisation and discharge, patients are likely to forget what has been explained to them \parencite{Roeung2024}. They could include contact numbers for questions they may have, the frequency and times to take medication, do's and don'ts and a brief description of the medication and what it is doing to aid in their recovery or maintenance of their condition. This can act as wrap-around support for a patient so that when they are discharged, they can feel well-prepared to manage their medication themselves.

%---------
%Word count
%---------
\vfill
\begin{center}
	\large Word Count: 2074
\end{center}
%---------
%Print  Bibliography
%---------
\pagebreak
\doublespacing
\printbibliography
\end{document}
